
\documentclass[a4paper,12pt]{article}

%%% Работа с русским языком
\usepackage{cmap}					% поиск в PDF
\usepackage{mathtext} 				% русские буквы в формулах
\usepackage[T2A]{fontenc}			% кодировка
\usepackage[utf8]{inputenc}			% кодировка исходного текста
\usepackage[english,russian]{babel}	% локализация и переносы

%%% Дополнительная работа с математикой
\usepackage{amsmath,amsfonts,amssymb,amsthm,mathtools} % AMS
\usepackage{icomma} % "Умная" запятая: $0,2$ --- число, $0, 2$ --- перечисление

%% Номера формул
%\mathtoolsset{showonlyrefs=true} % Показывать номера только у тех формул, на которые есть \eqref{} в тексте.
%\usepackage{leqno} % Нумерация формул слева

%% Свои команды
\DeclareMathOperator{\sgn}{\mathop{sgn}}

%% Перенос знаков в формулах (по Львовскому)
\newcommand*{\hm}[1]{#1\nobreak\discretionary{}
{\hbox{$\mathsurround=0pt #1$}}{}}

%%% Работа с картинками
\usepackage{graphicx}  % Для вставки рисунков
\graphicspath{{images/}{images2/}}  % папки с картинками
\setlength\fboxsep{3pt} % Отступ рамки \fbox{} от рисунка
\setlength\fboxrule{1pt} % Толщина линий рамки \fbox{}
\usepackage{wrapfig} % Обтекание рисунков текстом

%%% Работа с таблицами
\usepackage{array,tabularx,tabulary,booktabs} % Дополнительная работа с таблицами
\usepackage{longtable}  % Длинные таблицы
\usepackage{multirow} % Слияние строк в таблице

%%% Теоремы
\theoremstyle{plain} % Это стиль по умолчанию, его можно не переопределять.
\newtheorem{theorem}{Теорема}[section]
\newtheorem{proposition}[theorem]{Утверждение}

\theoremstyle{definition} % "Определение"
\newtheorem{corollary}{Следствие}[theorem]
\newtheorem{problem}{Задача}[section]

\theoremstyle{remark} % "Примечание"
\newtheorem*{nonum}{Решение}

%%% Программирование
\usepackage{etoolbox} % логические операторы

%%% Заголовок
\author{\LaTeX{}}
\title{Счетчики и макрокоманды}
\date{\today}

\begin{document} % конец преамбулы, начало документа

\maketitle

\section{Теоремы}

\begin{theorem}[Простое равенство]\label{theorem1}
	$2+2=4$
\end{theorem}

Смотри теорему \ref{theorem1} на стр. \pageref{theorem1}.

\begin{proposition}
	$3\times 3 = 9$
\end{proposition}

\begin{corollary}
	$9/3 = 3$
\end{corollary}

\begin{nonum}
	Всё что угодно.
\end{nonum}

\section{Новые команды}

\newcommand{\nw}{\ensuremath{\succcurlyeq}}

\begin{equation}\label{pref}
	x \nw y
\end{equation}

Предпочтение \nw\ является полным.

\newcommand{\str}[1]{%
на стр. \pageref{#1}%
}

\newcommand{\qwerty}[2][X]{
\begin{equation}
	#1 \nw #2
\end{equation}
}

\qwerty{Y}

$x\ge y$

\renewcommand{\ge}{\geqslant}

$x\ge y$

\section{Счетчики}

\newcounter{nc}[section]
\arabic{nc}

\setcounter{nc}{5}
\arabic{nc}

\Roman{nc}

\alph{nc}

\Asbuk{nc}

\arabic{section}

\renewcommand{\thesection}{\Asbuk{section}}
\newcommand{\z}[1]{%

\addtocounter{nc}{1}
Задача \thesection.\arabic{nc}. #1%
}

\z{Текст задачи.}

\z{Problem two}

\setcounter{section}{0}

\section{Окружения}

\newenvironment{zadacha}[1]{

\addtocounter{nc}{1}%
Задача \thesection.\arabic{nc}. <<#1>> %
}{\vspace{1cm}}

\begin{zadacha}{Главная задача}
Текст задачи.
\end{zadacha}

\section{etoolbox}

\newbool{answers}
\booltrue{answers}
\renewcommand{\z}[2][Нет ответа.]{%

\addtocounter{nc}{1}
Задача \thesection.\arabic{nc}. #2%

\ifbool{answers}{Ответ. #1}{}
}

\z[6]{Сколько будет $2+4$?}


\end{document} % конец документа
